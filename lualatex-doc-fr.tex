% lualatex-doc-fr: un guide touristique de LuaLaTeX
%
% Écrit par Manuel Pégourié-Gonnard <mpg@elzevir.fr>, 2010-2013.
% Adaptation française: Jérémy Just <jeremy@jejust.fr>, 2013-2014.
%
% Distributed under the terms of the GNU free documentation licence:
%   http://www.gnu.org/licenses/fdl.html
% without any invariant section or cover text.

\documentclass{lltxdoc}
  \selectlanguage{french}

\title{Un guide touristique de \lualatex\thanks{Traduit en français par Jérémy Just \email{jeremy@jejust.fr}}}
\author{Manuel Pégourié-Gonnard \email{mpg@elzevir.fr}}
%\date{\today}
\date{5 mai 2013}

\begin{document}

\maketitle

\begin{abstract}
  Ce document se veut être un guide touristique du nouveau monde de
  \lualatex.\footnote{bien que centré sur LuaLaTeX, il inclut également des
    informations utiles sur \luatex utilisé avec le format Plain.} Le public
  visé va des nouveaux venus (ayant une connaissance pratique du \latex
  conventionnel) aux développeurs de packages. Ce guide se veut exhaustif
  dans le sens suivant: il contient des pointeurs vers toutes les sources
  pertinentes, rassemble des informations qui sont sinon dispersées et
  ajoute des éléments d'introduction.

  Vos commentaires et suggestions d'améliorations sont les bienvenus.
  Ce document est un travail en cours; merci pour votre bienveillance et
  votre patience.
\end{abstract}

\vspace{\stretch{1}}
\setcounter{tocdepth}{2}
\listoftoc*{toc}
\vspace*{\stretch{2}}
\clearpage

\section{Introduction}\label{intro}

\subsection{Qu'est-ce que \lualatex?}\label{what}

Pour répondre à cette question, nous devons préciser un détail sur le monde \tex
que vous pouvez habituellement négliger: la différence entre un \emph{moteur}
et un \emph{format}. Un moteur est un programme informatique réel, tandis
qu'un format est un ensemble de macros exécutées par un moteur, et généralement
préchargé lorsque le moteur est invoqué sous un nom spécifique.

En fait, un format est plus ou moins comme une classe de document ou un paquetage,
sauf qu'il est associé à un nom de commande particulier. Imaginez qu'il existe une
commande \cmd{latex-article} qui ferait la même chose que \cmd{latex}, sauf que
vous n'auriez pas besoin de dire ©\documentclass{article}© au début de votre fichier.
De même, dans les distributions actuelles, la commande \cmd{pdflatex} est la même que
la commande \cmd{pdftex}, sauf que vous n'avez pas besoin de mettre les instructions
pour charger \latex au début de votre fichier source. C'est pratique, et légèrement
plus efficace aussi.

Les formats sont une belle invention car ils permettent d'implémenter des commandes
puissantes, en utilisant les outils de base fournis par le moteur. Cependant, la
puissance du format reste limitée par l'ensemble des outils du moteur, c'est pourquoi
les gens ont commencé à développer des moteurs plus puissants afin que d'autres
personnes puissent mettre en \oe uvre des formats (ou des packages) encore plus
puissants. Les moteurs les plus connus actuellement (à l'exception du \tex original)
sont \pdftex, \xetex et \luatex.

Pour compliquer encore le tableau, le moteur \tex original ne produisait que des
fichiers~DVI, alors que ses successeurs peuvent (aussi) produire des fichiers~PDF.
Chaque commande de votre système correspond à un moteur particulier avec un format
particulier et un mode de sortie particulier. Le tableau suivant résume cela:
les lignes indiquent le format, les colonnes le moteur, et dans chaque case,
la première ligne est la commande pour ce moteur avec ce format en mode~DVI,
et la seconde en mode~PDF.

\begin{center}
  \newcommand*\cell [2] {%
    \parbox{6em}{\centering\leavevmode\color{code}\ttfamily
      \strut\maybe{#1} \\ \strut\maybe{#2}}}
  \makeatletter
  \newcommand*\maybe [1] {%
    \@ifmtarg{#1} {\textcolor{gray}{\normalfont (aucun)}} {#1}}
  \begin{tabular}{l|cccc}
                & \tex & \pdftex & \xetex & \luatex
    \\ \hline
    Plain
                &  \cell{tex}{}
                &  \cell{etex}{pdftex}
                &  \cell{}{xetex}
                &  \cell{dviluatex}{luatex}
    \\ \hline
    \latex
                &  \cell{}{}
                &  \cell{latex}{pdflatex}
                &  \cell{}{xelatex}
                &  \cell{dvilualatex}{lualatex}
    \\
  \end{tabular}
\end{center}

Nous pouvons maintenant répondre à la question posée plus haut: \lualatex est
le moteur \luatex avec le format \latex. Cette réponse n'est pas très satisfaisante
si vous ne savez pas ce qu'est \luatex (et peut-être \latex).

\medskip

Commençons par ce que vous savez sans doute déjà: dans le monde \tex au sens large,
\latex est le cadre général dans lequel les documents commencent par
©\documentclass©, les paquets sont chargés par ©\usepackage©, les polices sont
sélectionnées de manière intelligente (de sorte que vous puissiez passer en gras
tout en préservant l'italique), les pages sont construites à l'aide d'algorithmes
compliqués comprenant la prise en charge des en-têtes, des pieds de page, des
notes de bas de page, des notes de marge, des flottants, etc. Tout cela ne change
pas avec \lualatex, mais de nouveaux paquets plus puissants sont disponibles
pour améliorer le fonctionnement de certaines parties du système.


Alors, qu'est-ce que \luatex? Version courte: le moteur \tex le plus
populaire du moment!
Version longue: c'est le successeur désigné de pdfTeX et il inclut
toutes ses fonctionnalités principales: génération directe de fichiers~PDF
avec support des fonctionnalités PDF avancées et améliorations
micro-typographiques des algorithmes typographiques \tex.
Les principales nouveautés de \luatex sont:
\begin{enumerate}
  \item Support natif d'Unicode, la norme moderne de classement et
    d'encodage des caractères, supportant tous les caractères du monde,
    de l'anglais au chinois traditionnel en passant par l'arabe,
    y compris de nombreux symboles mathématiques (ou symboles
    spécifiques d'autres domaines).
  \item Intégration de Lua comme langage de script embarqué
    (voir section~\ref{luaintex} pour plus de détails).
  \item Une multitude de merveilleuses bibliothèques Lua, notamment:
    \begin{itemize}
      \item ©fontloader©, prenant en charge les formats de polices modernes
        tels que TrueType et OpenType;
      \item ©font©, permettant une manipulation avancée des polices à partir
        du document;
      \item ©mplib©, une version embarquée du programme graphique MetaPost;
      \item ©callback©, qui permet d'accéder à des parties du moteur \tex
        qui étaient auparavant inaccessibles au programmeur;
      \item des bibliothèques utilitaires pour la manipulation d'images,
        de fichiers~PDF, etc.
    \end{itemize}
\end{enumerate}
Certaines de ces fonctionnalités, comme la prise en charge d'Unicode, ont un
impact direct sur tous les documents, tandis que d'autres fournissent simplement
des outils que les auteurs de paquets utiliseront pour vous fournir des commandes
plus puissantes et autres améliorations.


\subsection{Passage de \latex à \lualatex}\label{switch}

Comme l'explique la section précédente, \lualatex{} est en grande partie comme \latex,
avec quelques différences, et des paquets et outils plus puissants disponibles.
Nous présentons ici le minimum absolu que vous devez savoir pour produire
un document avec \lualatex, tandis que le reste du document fournit
plus de détails sur les paquets disponibles.


Il n'y a que trois différences:
\begin{enumerate}
  \item Ne chargez pas \pf{inputenc}, encodez simplement votre source en UTF-8.
  \item Ne chargez pas \pf{fontenc} ni \pf{textcomp}, mais chargez \pf{fontspec}
     à la place.
  \item \pf{babel} fonctionne avec \lualatex{} mais vous pouvez charger \pf{polyglossia}
     à la place.
  \item N'utilisez pas de paquet qui change les polices, mais utilisez les commandes
    de \pf{fontspec} à la place.
\end{enumerate}
Ainsi, vous n'avez qu'à vous familiariser avec \pf{fontspec}, ce qui est facile:
sélectionnez la police principale (avec empattement) avec ©\setmainfont©,
la police sans empattement avec ©\setsansfont© et la police à chasse fixe
(style machine à écrire) avec ©\setmonofont©. L'argument de ces commandes est
le petit nom de la police, lisible par un humain, par exemple ©Latin Modern Roman©
(et non ©ec-lmr10©). Vous voudrez probablement utiliser ©\defaultfontfeatures{Ligatures=TeX}©
avant ces commandes pour que les substitutions \tex habituelles
(comme ©---© pour un tiret cadratin) fonctionnent.

La bonne nouvelle est que vous pouvez accéder directement à n'importe quelle
police de votre système d'exploitation (en plus de celles de votre distribution \tex),
y compris les polices TrueType et OpenType, et avoir accès à leurs fonctionnalités
les plus avancées. Cela signifie qu'il est désormais facile d'installer n'importe
quelle police moderne que vous pouvez télécharger ou acheter auprès d'un éditeur,
de les utiliser avec \lualatex{} et de bénéficier de tout leur potentiel.

Passons maintenant aux mauvaises nouvelles: il n'est pas toujours facile d'obtenir
une liste de toutes les polices disponibles. Sous Windows avec \texlive, l'outil
de ligne de commande \cmd{fc-list} les liste toutes, mais n'est pas très convivial.
Sous Mac OS~X, l'application \emph{Fontbook} liste les polices de votre système,
mais pas celles de votre distribution \tex. Même chose avec \cmd{fc-list} sous Linux.
Autre mauvaise nouvelle: il n'est pas facile d'accéder à vos anciennes polices
de cette manière. Heureusement, progressivement (et rapidement), de plus en plus
de polices sont disponibles dans des formats modernes.

Soit dit en passant, le contenu de cette section jusqu'à présent vaut aussi pour
\xelatex, c'est-à-dire \latex sur \xetex. En effet, \xetex partage deux des
caractéristiques essentielles de \luatex: l'Unicode natif et le support des formats
de polices modernes (en revance, il n'a pas les autres caractéristiques de \luatex;
mais actuellement, il est considéré comme plus stable). Bien que leurs implémentations
concernant les polices de caractères soient très différentes, \pf{fontspec} parvient
à offrir une interface de police pratiquement unifiée pour \xelatex et \lualatex.

\medskip

Ainsi, pour bénéficier des nouvelles fonctionnalités de \luatex, vous devez
renoncer à un peu de l'ancien monde, à savoir les polices qui ne sont pas
disponibles dans un format moderne (ainsi qu'à la liberté d'encoder votre source
comme bon vous semble, mais UTF-8 est tellement supérieur aux autres encodages
que vous ne perdez quasiment rien au change). Le package \pk{luainputenc}
fournit des solutions de transition qui vous permettent de retrouver certains
anciens comportemenbts\footnote{Bien que son nom suggère qu'il ne s'occupe
  que des encodages d'entrée, l'implémentation de l'encodage des polices
  en \latex implique que ce package est nécessaire (et fonctionne)
  également pour utiliser les anciennes polices.}, peut-être au prix
de la perte du support réel d'Unicode.

En gros, c'est tout ce que vous devez savoir pour commencer à produire des
documents avec \lualatex. Je vous recommande de jeter un coup d'oeil au manuel
de \pf{fontspec} et d'essayer de compiler vous-même un petit document
en utilisant des polices amusantes.
Vous pourrez ensuite parcourir le reste de ce document comme bon vous semble.
La section~\ref{workornot} liste toutes les autres différences que je connais
entre \latex conventionnel et \lualatex.


\subsection{Une introduction à Lua-dans-\tex}\label{luaintex}

Lua est un petit langage, plutôt bien pensé, bien moins surprenant que \tex
en tant que langage de programmation, et beaucoup plus facile à apprendre que lui.
La référence essentielle est l'excellent livre \emph{Programming in Lua},
dont la première édition est \href{http://www.lua.org/pil/}{disponible gratuitement en ligne}
(en anglais). Pour commencer rapidement, je vous recommande de lire les
chapitres~1 à~5 et de jeter un coup d'\oe il à la partie~3. Notez que toutes
les bibliothèques mentionnées dans le chapitre~3 sont incluses dans \luatex,
mais que la bibliothèque ©os© est restreinte pour des raisons de sécurité.

En fonction de vos connaissances en matière de programmation, vous serez
peut-être directement intéressé par le reste de la partie~1 et la partie~2,
qui présentent des fonctionnalités plus avancées du langage, mais la partie~4
est inutile dans un contexte de \luatex, à moins bien sûr que vous ne vouliez
modifier \luatex lui-même. Enfin, le manuel de référence de Lua est
\href{http://www.lua.org/manual/5.2/}{disponible en ligne} et est accompagné d'un index très pratique.

\medskip

Passons maintenant à l'utilisation de Lua \emph{dans} \luatex. La principale
façon d'exécuter du code Lua à partir de \tex est la commande ©\directlua©,
qui prend du code Lua arbitraire comme argument. Inversement, vous pouvez
passer des informations de Lua à \tex avec la commande ©tex.sprint©\footnote{Dans
  ce nom, \og{}sprint\gf{} signifie \og{}\emph{string print}\fg{}
  (\og{}imprimer une chaîne\fg{}), et non \og{}aller très vite"\fg{}!}.
Par exemple,
\begin{Verbatim}
  approximation standard de $\pi = \directlua{tex.sprint(math.pi)}$
\end{Verbatim}
imprime \og{}approximation standard de $\pi = \directlua{tex.sprint(math.pi)}$\fg dans votre document. Vous voyez comme il est facile de mélanger du \tex et du Lua?

Actually, there are a few gotchas. Let's look at the Lua to \tex way first,
it's the simplest (since it's more Lua than \tex). If you look at the \luatex
manual, you'll see there is another function with a simpler name, ©tex.print©,
to pass information this way. It works by virtually inserting a full line into
your \tex source, whose contents are its argument. In case you didn't know,
\tex does many nasty\footnote{Okay, these are usually nice and helpful
  actions, but in this case they are most probably unexpected so I call them
  nasty.} things with full lines of the source:
ignoring spaces at the beginning and end of line and appending an end-of-line
character. Most of the time, you don't want this to happen, so I recommend
using ©tex.sprint© which virtually inserts its argument in the current line,
so is much more predictable.

En fait, il y a quelques astuces à noter. Regardons d'abord le passage de Lua vers Tex, c'est le plus simple (puisqu'il s'agit davantage de Lua que de Tex). Si vous consultez le manuel LuaTeX, vous verrez qu'il existe une autre fonction avec un nom plus simple, ©tex.print©, pour faire passer des informations dans ce sens. Elle fonctionne en insérant virtuellement une ligne complète dans votre source TeX, dont le contenu est son argument. Au cas où vous ne le sauriez pas, TeX fait beaucoup de choses désagréables (d'accord, ce sont généralement des actions utiles et bien intentionnées, mais dans ce cas présent, elles sont inattendues, donc je les appelle désagréables) avec les lignes complètes de la source: ignorer les espaces en début et en fin de ligne et ajouter un caractère de fin de ligne. La plupart du temps, vous ne voulez pas que cela se produise, donc je recommande d'utiliser ©tex.sprint© qui insère simplement son argument dans la ligne courante, et donne un résultat plus prévisible.
%%

If you're enough of a \tex{}nician to know about catcodes, you'll be happy to
know that ©tex.print© and its variants give you nearly full control over the
catcodes used for tokenizing the argument, since you can specify a catcode
table as the first argument. You'll probably want to learn about catcode
tables (currently~2.7.6 in the \luatex manual) before you're fully happy. If
you don't know about catcodes, just skip this paragraph.\footnote{Erf, too
  late, you already read it.}

Si vous êtes suffisamment bon TeXnicien pour connaître les catcodes, vous serez heureux d'apprendre que ©tex.print© et ses variantes vous donnent un contrôle presque total sur les catcodes utilisés pour tokeniser l'argument, puisque vous pouvez spécifier une table de catcodes comme premier argument. Les tables de catcodes sont présentées à la section~2.7.6 dans le manuel LuaTeX (dans la version actuelle), vous avez sans doute intérêt à y jeter un \oe il. Si vous ne connaissez pas les catcodes, passez ce paragraphe.\footnote{Arf, trop tard, vous l'avez déjà lu\dots}
%%

\medskip

Let's look at ©\directlua© now. To get an idea about how it works, imagine that
it's a ©\write© command, but it writes only to a virtual file and immediately
arranges for this file to be fed to the Lua interpreter. On the Lua end, the
consequence is that each argument of a ©\directlua© command has its own scope:
local variables from one will not be visible to the other. (Which is rather
sane, but always good to know.)

Regardons maintenant ©\directlua©. Pour vous faire une idée de son fonctionnement, imaginez qu'il s'agit d'une commande ©\write©, mais qu'elle écrit uniquement dans un fichier virtuel et s'arrange pour que ce fichier soit immédiatement transmis à l'interpréteur Lua. Du côté de Lua, la conséquence est que chaque argument d'une commande ©\directlua© a sa propre portée: les variables définies localement dans un argument ne seront pas visibles par le suivant (ce qui est plutôt sain, mais toujours bon à savoir).
%%

Now, the major gotcha is that before being fed to the Lua interpreter, the
argument is first read and tokenised by \tex, then fully expanded and turned
back into a plain string. Being read by \tex has several consequences. One of
them is that end of lines are turned into spaces, so the Lua interpreter sees
only one (long) line of input. Since Lua is a free-form language, it doesn't
usually matter, but it does if you use comments:

Maintenant, le problème majeur est qu'avant d'être transmis à l'interpréteur Lua, l'argument est d'abord lu et tokénisé par TeX, puis entièrement développé et transformé en une chaîne de caractères ordinaire. La lecture par TeX a plusieurs conséquences. L'une d'entre elles est que les fins de lignes sont transformées en espaces, de sorte que l'interprète Lua ne voit qu'une (longue) ligne d'entrée. Comme Lua est un langage de forme libre, cela n'a généralement pas d'importance, sauf si vous utilisez des commentaires:
%%
\begin{Verbatim}
  \directlua{une_fonction()
    -- un commentaire
    une_autre_fonction()}
\end{Verbatim}
ne fera pas ce que vous attendez probablement: ©une_autre_fonction()© sera
considéré comme faisant partie du commentaire (tout est mis sur une seule
ligne, ne l'oubliez pas).
%% (already merged)

Another consequence of being read by \tex is that successive spaces are merged
into one space, and \tex comments are discarded. So, here is a surprisingly
correct version of the previous example.

Une autre conséquence de la lecture par TeX est que les espaces successives sont fusionnés en une unique espace, et que les commentaires TeX sont éliminés. Voici donc une version correcte de l'exemple précédent, de façon surprenante:
%%

\begin{Verbatim}
  \directlua{une_fonction()
    % un commentaire
    une_autre_fonction()}
\end{Verbatim}
It is also worth noticing that, since the argument basically is inside a
©\write©, it's in expansion-only context. If you don't know what it means, let
me say that expansion issues are mostly what makes \tex programming so
difficult to avoid expanding further on that matter.

Il convient également de noter que, puisque l'argument se trouve essentiellement à l'intérieur d'un ©\write©, il se trouve dans un contexte d'expansion uniquement. Si vous ne savez pas ce que cela signifie, laissez-moi seulement vous dire que les problèmes d'expansion sont ce qui rend la programmation TeX si difficile et qu'il vaut mieux éviter de développer davantage cette question aujourd'hui.
%%

\medskip

I'm sorry if the last three paragraphs were a bit \tex{}nical in nature but
I thought you had to know. To reward you for staying with me, here is a
debugging trick. Put the following code near the beginning of your document:

Je vous prie de m'excuser si les trois derniers paragraphes ont été un peu \tex{}niques mais j'ai préféré vous prévenir de ces pièges. Pour vous récompenser d'être resté avec moi, voici une astuce de débogage. Collez le code suivant au début de votre document:
%%
\begin{Verbatim}
  \newwrite\luadebug
  \immediate\openout\luadebug luadebug.lua
  \AtEndDocument{\immediate\closeout\luadebug}
  \newcommand\directluadebug{\immediate\write\luadebug}
\end{Verbatim}
Then, when you have a hard time understanding why a particular call to
©\directlua© doesn't do what you expect, replace this instance of the command
with ©\directluadebug©, compile as usual and look in the file
\file{luadebug.lua} produced what the Lua interpreter actually read.

Ensuite, lorsque vous aurez du mal à comprendre pourquoi un appel particulier à ©\directlua© ne fait pas ce que vous attendez, remplacez cette instance de la commande par ©\directluadebug©, compilez comme d'habitude et regardez dans le fichier \file{luadebug.lua} ce que l'interpréteur Lua a réellement lu.
%%

The \pk{luacode} package provides commands and environments that help to
varying degrees with some of these problems. However, for everything but
trivial pieces of Lua code, it is wiser to use an external file containing
only Lua code defining functions, then load it and use its functions. For
example:

Le package "luacode" fournit des commandes et des environnements qui aident de différentes façons à résoudre certains de ces problèmes. Cependant, dès que le code Lua utilisé n'est plus trivial, il est plus sage d'utiliser un fichier externe contenant uniquement du code Lua définissant des fonctions, puis de le charger et d'appeler ses fonctions depuis le document LuaTeX. Par exemple:
%%
\begin{Verbatim}
  \directlua{dofile("mes-functions-lua.lua")}
  \newcommand*{\macrogeniale}[2]{%
    \directlua{ma_fonction_geniale("\luatexluaescapestring{#1}", #2)}}
\end{Verbatim}
The example assumes that ©my_great_function© is defined in
©my-lua-functions.lua© and takes a string and a number as arguments. Notice
that we carefully use the ©\luatexluaescapestring© primitives on the string
argument to escape any backslash or double-quote it might contain and which
would confuse the Lua parser.\footnote{If you ever used SQL then the concept
  of escaping strings is hopefully not new to you.}

L'exemple suppose que ©ma_fonction_geniale© est définie dans ©mes-fonctions-lua.lua© et prend une chaîne de caractères et un nombre comme arguments. Remarquez que nous utilisons soigneusement les primitives ©\luatexluaescapestring© sur l'argument chaîne de caractères afin d'échapper à toute barre oblique inverse ou double-citation qu'il pourrait contenir et qui pourrait perturber l'analyseur syntaxique Lua. (Si vous avez déjà utilisé SQL, le concept d'échappement des chaînes de caractères n'est pas nouveau pour vous).
%%

\medskip

That's all concerning Lua in \tex. If you're wondering why
©\luatexluaescapestring© has such a long and silly name, you might want to
read the next section.

C'est tout pour ce qui concerne Lua dans TeX. Maintenant, si vous vous demandez pourquoi ©\luatexluaescapestring© a un nom aussi ridiculement long, lisez la section suivante.
%%

\subsection{Autres choses à savoir}\label{things}

Just in case it isn't obvious, the \luatex manual, \file{luatexref-t.pdf}, is
a great source of information about \luatex and you'll probably want to
consult it at some point (though it is a bit arid and technical).

Avant toute chose, mentionnons que le manuel LuaTeX, "luatexref-t.pdf", est une excellente source d'informations sur LuaTeX et vous voudrez probablement le consulter à un moment ou à un autre (bien qu'il soit un peu aride et technique).
%%

It is important to know that the name of the new primitives of \luatex as you
read them in the manual are not the actual names you'll be able to use in
\lualatex. To prevent clashes with existing macro names, all new primitives
have been prefixed with ©\luatex© unless they already start with it, so
©\luaescapestring© becomes ©\luatexluaescapetring© while ©\luatexversion©
remains ©\luatexversion©. The rationale is detailed in section~\ref{formats}.

Il est important de savoir que les noms des nouvelles primitives de LuaTeX tels que vous les lisez dans le manuel ne sont pas les noms réels que vous pourrez utiliser dans LuaLaTeX. Pour éviter les conflits avec les noms de macros existants, toutes les nouvelles primitives ont été préfixées par ©\luatex©, à moins qu'elles ne commencent déjà par ce nom. Ainsi, ©\luaescapestring© devient ©\luatexluaescapetring©, tandis que ©\luatexversion© reste ©\luatexversion©. Le raisonnement est détaillé dans la section 4.
%%

\medskip

Oh, and by the way, did I mention that \luatex is in beta and version 1.0 is
expected in spring 2014? You can learn more on the roadmap page of
\href{http://luatex.org/}{the \luatex site}. Stable betas are released
regularly and are included in \texlive since 2008 and \miktex since 2.9.

Oh, et au fait, ai-je mentionné que LuaTeX est en version bêta et que la version 1.0 est attendue au printemps 2014 ? Vous pouvez en apprendre davantage sur la feuille de route présentée sur site web de LuaTeX. Des versions bêta stables sont publiées régulièrement et sont incluses dans TeX Live depuis 2008, et dans MikTeX depuis 2.9.
%%

Not surprisingly, support for \luatex in \latex is shiny new, which means it
may be full of (shiny) bugs, and things may change at any point. You might
want to keep your \tex distribution very up-to-date\footnote{For \texlive,
  consider using the complementary
  \href{http://tlcontrib.metatex.org/} {tlcontrib} repository.} and also avoid
using \lualatex for critical documents at least for some time.

Le support de LuaTeX dans LaTeX est tout nouveau, ce qui signifie qu'il peut être bourré de bugs et que les choses peuvent encore changer à tout moment. Il est donc important de garder votre distribution TeX à jour (pour avoir les corrections de bugs) et éviter d'utiliser LuaLaTeX pour des documents critiques, au moins pendant un certain temps (pour ne pas rencontrer un nouveau bug au mauvais moment).
%%

As a general rule, this guide documents things as they are at the time it is
written or updated, without keeping track of changes. Hopefully, you'll update
your distribution as a whole so that you always get matching versions of this
guide and the packages, formats and engine it describes.

En règle générale, ce guide documente les choses telles qu'elles sont au moment où il est écrit ou mis à jour, sans tenir compte des changements. Nous espérons que vous mettrez à jour votre distribution dans son ensemble afin d'obtenir toujours des versions correspondantes de ce guide et des paquets, formats et moteur qu'il décrit.
%%


\selectlanguage{english}
\section{Essential packages and practices}\label{essential}

This section presents the packages you'll probably want to always load as a
user, or that you should absolutely know about as a developer.

Cette section présente les packages que vous voudrez sans doute toujours charger en tant qu'utilisateur, ou que vous devez absolument connaître en tant que développeur.
%%

\subsection{Niveau utilisateur}

\pkdesc{fontspec}{\WSPR}{\xetex, \luatex}{\latex}{%
  macros/latex/contrib/fontspec/}[https://github.com/wspr/fontspec/]
Nice interface to font management, well-integrated in to the \latex font
selection scheme. Already presented in the previous section.

Interface conviviale pour la gestion des polices, bien intégrée dans le schéma de sélection des polices de LaTeX. Déjà présenté dans la section précédente.
%%

\pkdesc{polyglossia}{\FC \& \AR}{\xetex, \luatex}{\latex}{%
  macros/latex/contrib/polyglossia/}[https://github.com/reutenauer/polyglossia/]
A simple and modern replacement for Babel, working in synergy with \pk{fontspec}.

Un remplacement simple et moderne de Babel, travaillant main dans la main avec "fontspec".
%%

\subsection{Niveau développeur}

\subsubsection{Conventions de nommage}

On the \tex end, control sequences starting with ©\luatex© are reserved for
primitives. It is strongly recommended that you do \emph{not} define any such
control sequence, to prevent name clashes with future versions of \luatex. If
you want to a emphasize that a macro is specific to \luatex, we recommend that
you use the ©\lua© prefix (without a following ©tex©) instead. It is okay to
use the ©\luatex@© prefix for internal macros, since primitive names never
contain ©@©, but it might be confusing. Moreover, you're already using a
unique prefix for internal macros in all of your packages, aren't you?

Du côté de TeX, les séquences de contrôle commençant par "\luatex" sont réservées aux primitives. Il est fortement recommandé de ne pas définir de telles séquences de contrôle, afin d'éviter les conflits de noms avec les futures versions de LuaTeX. Si vous souhaitez souligner qu'une macro est spécifique à LuaTeX, nous vous recommandons d'utiliser le préfixe "\lua" (sans le "tex" suivant). Il est possible d'utiliser le préfixe "\luatex@" pour les macros internes, puisque les noms des primitives ne contiennent jamais "@", mais cela peut prêter à confusion. De plus, vous utilisez déjà un préfixe unique pour les macros internes dans tous vos paquets, n'est-ce pas?
%%

On the Lua end, please keep the global namespace as clean as possible. That
is, use a table ©mypackage© and put all your public functions and objects in
this table. You might want to avoid using Lua's deprecated \code{module()}.
Other strategies for Lua module management are discussed in
\href{http://www.lua.org/pil/15.html}{chapter~15 of \emph{Programming in
Lua}}, and examples are given in \file{luatexbase-modutils.pdf}. Also, it
is a good and necessary practice to use ©local© for your internal variables and
functions. Finally, to avoid clashes with future versions of \luatex, it is
necessary to avoid modifying the namespaces of \luatex's default libraries.

En ce qui concerne Lua, veuillez garder l'espace de noms global aussi propre que possible. En d'autres termes, utilisez une table "mypackage" et placez toutes vos fonctions et objets publics dans cette table. Vous pouvez aussi éviter d'utiliser la fonction "module()" de Lua, qui est obsolète. D'autres stratégies pour la gestion des modules Lua sont discutées dans le chapitre 15 de "Programming in Lua", et des exemples sont donnés dans "luatexbase-modutils.pdf". De plus, c'est une saine habitude d'utiliser local pour vos variables et fonctions internes. Enfin, pour éviter tout conflit avec les futures versions de LuaTeX, il est nécessaire d'éviter de modifier les espaces de noms des bibliothèques par défaut de LuaTeX.
%%

\subsubsection{Détection du moteur et du mode}\label{detect}

Various packages allow to detect the engine currently processing the document.

Plusieurs paquets permettent de détecter le moteur qui traite actuellement le document.
%%

\pkdesc{ifluatex}{\HO}{all}{\latex, Plain}{%
  macros/latex/contrib/oberdiek/}
Provides ©\ifluatex© and makes sure ©\luatexversion© is available.

Fournit "\ifluatex" et s'assure que "\luatexversion" est disponible.
%%

\pkdesc{iftex}{\VK}{all}{\latex, Plain}{%
  macros/latex/contrib/iftex/}[http://bitbucket.org/vafa/iftex]
Provides ©\ifPDFTeX©, ©\ifXeTeX©, ©\ifLuaTeX© and corresponding ©\Require©
commands.

Fournit les commandes "\ifPDFTeX", "\ifXeTeX", "\ifLuaTeX" et les commandes "\Require" correspondantes.
%%

\pkdesc{expl3}{The \LaTeX3 Project}{all}{\latex}{%
  macros/latex/contrib/expl3/}[http://www.latex-project.org/code.html]
Amongst \emph{many} other things, provides ©\luatex_if_engine:TF©,
©\xetex_if_engine:TF© and their variants.

Fournit, entre autres, "\luatex_if_engine:TF", "\xetex_if_engine:TF" et leurs variantes.
%%

\pkdesc{ifpdf}{\HO}{all}{\latex, Plain}{%
  macros/latex/contrib/oberdiek/}
Provides ©\ifpdf© switch. \luatex, like \pdftex, can produce either PDF or DVI
output; the later is not very useful with \luatex as it doesn't support any
advanced feature such as Unicode and modern font formats. The ©\ifpdf© switch
is true if and only if you are running \pdftex-or-\luatex in PDF mode (note
that this doesn't include \xetex, whose support for PDF is different).

Fournit le commutateur "\ifpdf". LuaTeX, comme pdfTeX, peut produire une sortie PDF ou DVI ; cette dernière n'est pas très utile avec LuaTeX car elle ne supporte aucune fonctionnalité avancée telle que l'Unicode et les formats de police modernes. Le commutateur "\ifpdf" est vrai si et seulement si vous exécutez pdfTEX-ou-LuaTEX en mode PDF (notez que cela n'inclut pas XeTeX, dont le support pour PDF est différent).
%%

\subsubsection{Ressources de base}

\pkdesc{luatexbase}{\ER, \MPG \& \PHG}{\luatex}{\latex, Plain}{%
  macros/luatex/generic/luatexbase/}[https://github.com/lualatex/luatexbase]
The Plain and \latex formats provide macros to manage \tex basic resources,
such as count or box registers. \luatex introduces new resources that need to
be shared gracefully by packages. This package provides the basic tools to
manage: the extended conventional \tex resources, catcode tables, attributes,
callbacks, Lua modules loading and identification. It also provides basic
tools to handle a few compatibility issues with older version of \luatex.

Les formats Plain et LaTeX fournissent des macros pour gérer les ressources de base de TeX, comme les compteurs ou les registres de boîtes. LuaTeX introduit de nouvelles ressources qui doivent être partagées intelligemment par les paquets. Ce paquetage fournit les outils de base pour gérer: les ressources TeX conventionnelles étendues, les tables de catcodes, les attributs, les callbacks, le chargement et l'identification des modules Lua. Il fournit également des outils de base pour gérer quelques problèmes de compatibilité avec les anciennes versions de LuaTeX.
%%

\note{Warning} This package is currently in conflict with the \pk{luatex}
package, since they both do almost the same thing. The respective package
authors are well aware of this situation and plan to somehow merge the two
packages in the near future, though the timeline is not clear.

Attention: Ce paquet est actuellement en conflit avec le paquet "luatex", puisqu'ils font quasiment la même chose. Les auteurs des deux packages sont bien conscients de cette situation et prévoient de les fusionner d'une manière ou d'une autre dans un avenir proche, bien que le calendrier ne soit pas encore fixé.
%%

\pkdesc{luatex}{\HO}{\luatex}{\latex, Plain}{%
  macros/latex/contrib/oberdiek/}
See the description of \pk{luatexbase} above. This package provides the same
core features except for callback management and Lua module identification.

Voir la description de luatexbase ci-dessus. Ce package fournit les mêmes fonctionnalités de base, à l'exception de la gestion des callbacks et de l'identification des modules Lua.
%%

\pkdesc{lualibs}{\ER \& \PHG}{\luatex}{Lua}{%
  macros/luatex/generic/lualibs/}[https://github.com/lualatex/lualibs]
Collection of Lua libraries and additions to the standard libraries; mostly
derived from the \context libraries. If you need a basic function that Lua
doesn't provide, check this package before rolling your own implementation.

Collection de bibliothèques Lua et d'ajouts aux bibliothèques standards ; principalement dérivées des bibliothèques ConTeXt. Si vous avez besoin d'une fonction de base que Lua ne fournit pas, consultez ce package avant de développer votre propre implémentation.
%%

\subsubsection{Gestion interne des polices de caractères}\label{fontint}

Those packages are loaded by \pk{fontspec} to handle some low-level font and
encoding issues. A normal user should only use \pk{fontspec}, but a developer
may need to know about them.


Ces packages sont chargés par "fontspec" pour gérer certaines polices de bas niveau et les problèmes d'encodage. Un utilisateur normal ne devrait utiliser que "fontspec", mais un développeur peut avoir besoin de les connaître.
%%

\pkdesc{luaotfload}{\ER, \KH \& \PHG}{\luatex}{\latex, Plain}{%
  macros/luatex/generic/luaotfload/}[https://github.com/lualatex/luaotfload]
Low-level OpenType font loading, adapted from the generic subset of \context.
Basically, it uses the ©fontloader© Lua library and the appropriate callbacks
to implement a syntax for the ©\font© primitive very similar to that of \xetex
and implement the corresponding font features. It also handles a font database
for transparent access to fonts from the system and the \tex distribution
either by family name or by file name, as well as a font cache for faster
loading.

Chargement de bas niveau des polices OpenType, adapté du code générique de ConTeXt. En gros, il utilise la bibliothèque Lua "fontloader" et les callbacks correspondants pour implémenter une syntaxe pour la primitive "\font" très similaire à celle de XeTeX et donner accès aux propriétés correspondantes des polices. Il gère également une base de données de polices pour un accès transparent aux polices du système et de la distribution TeX, soit par nom de famille, soit par nom de fichier, ainsi qu'un système de cache pour un chargement plus rapide des polices.
%%

\pkdesc{euenc}{\WSPR, \ER \& \KH}{\xetex, \luatex}{\latex}{%
  macros/latex/contrib/euenc/}[https://github.com/wspr/euenc]
Implements the EUx Unicode font encodings for \latex's \pf{fontenc} system.
Currently, \xelatex is using ©EU1© and \lualatex is using ©EU2©. Includes
definitions (\file{fd} files) for Latin Modern, the default font loaded by
\pk{fontspec}.

Implémente les encodages de polices Unicode "EUx" pour le système fontenc de LaTeX. Actuellement, XeLaTeX utilise "EU1" et LuaLaTeX utilise "EU2". Inclut les définitions (fichiers "fd") pour Latin Modern, la police chargée par défaut par "fontspec".
%%

To be precise, \pf{euenc} merely declares the encoding, but
doesn't provide definitions for LICR macros; this is done by loading
\pk{xunicode} with ©\UTFencname© defined to ©EU1© or ©EU2©, which
\pk{fontspec} does. The actual encodings are the same, but it is useful to
have distinct names so that different \file{fd} files can be used according to
the engine (which is actually the case with Latin Modern).

Pour être précis, "euenc" déclare simplement l'encodage, mais ne fournit pas de définitions pour les macros LICR ; ceci est fait en chargeant "xunicode" avec "\UTFencname" défini à "EU1" ou "EU2", ce que fait "fontspec". Les encodages réels sont les mêmes, mais il est utile d'avoir des noms distincts pour que différents fichiers "fd" puissent être utilisés selon le moteur (ce qui est en fait le cas avec Latin Modern).
%%

\section{Autres packages}\label{other}

Note that the packages are not listed in any particular order.

Notez que les paquets sont listés sans ordre particulier.
%%

\subsection{Niveau utilisateur}

\pkdesc{luatextra}{\ER \& \MPG}{\luatex}{\latex}{%
  macros/luatex/latex/luatextra/}[https://github.com/lualatex/luatextra]
Loads usual packages, currently \pk{fontspec}, \pk{luacode}, \pf{metalogo}
(commands for logos, including ©\LuaTeX© and ©\LuaLaTeX©), \pk{luatexbase},
\pk{lualibs}, \pf{fixltx2e} (fixes and enhancements for the \latex core).

Charge les packages habituels, actuellement "fontspec", "luacode", "metalogo" (commandes pour les logos, y compris "\LuaTeX" et "\LuaLaTeX"), "luatexbase", "lualibs", "fixltx2e" (corrections et améliorations pour le noyau LaTeX).
%%

\pkdesc{luacode}{\MPG}{\luatex}{\latex}{%
  macros/luatex/latex/luacode/}[https://github.com/lualatex/luacode]
Provides commands and macros that help including Lua code in a \tex source,
especially concerning special characters.

Fournit des commandes et des macros qui aident à inclure du code Lua dans un source TeX, en particulier concernant les caractères spéciaux.
%%

\pkdesc{luainputenc}{\ER \& \MPG}{\luatex, \xetex, \pdftex}{\latex}{%
  macros/luatex/latex/luainputenc/}[https://github.com/lualatex/luainputenc]
Helps compiling documents relying on legacy encodings (either in the source or
with the fonts). Already presented in the introduction. When running \xetex,
just loads \pf{xetex-inputenc}; under \pdftex, loads the standard
\pf{inputenc}.

Aide à la compilation de documents reposant sur des encodages anciens (soit dans le source, soit avec les polices). Déjà présenté dans l'introduction. Sous XeTeX, charge simplement "xetex-inputenc"; sous pdfTeX, charge l'"inputenc" standard.
%%

\pkdesc{luamplib}{\HH, \Taco \& \PHG}{\luatex}{\latex, Plain}{%
  macros/luatex/generic/luamplib/}[https://github.com/lualatex/luamplib]
Provides a nice interface for the ©mplib© Lua library that embeds metapost in
\luatex.

Fournit une interface conviviale pour la bibliothèque Lua mplib qui intègre metapost dans LuaTeX.
%%

\pkdesc{luacolor}{\HO}{\luatex}{\latex}{%
  macros/latex/contrib/oberdiek/}
Changes low-level color implementation to use \luatex attributes in place of
whatsits. This makes the implementation more robust and fixes strange bugs
such as wrong alignment when ©\color© happens at the beginning of a ©\vbox©.

Change l'implémentation de bas niveau des couleurs pour utiliser les attributs LuaTeX à la place des whatsits. Cela rend l'implémentation plus robuste et corrige des bugs bizarres tels qu'un mauvais alignement lorsque "\color" se trouve au début d'un "\vbox".
%%

\subsection{Niveau développeur}

\pkdesc{pdftexcmds}{\HO}{\luatex, \pdftex, \xetex}{\latex, Plain}{%
  macros/latex/contrib/oberdiek/}
Though \luatex is mostly a superset of \pdftex, a few utility primitives were
removed (those that are sort of superseded by Lua) or renamed. This package
provides them with consistent names across engines, including \xetex which
recently implemented some of these primitives, such as ©\strcmp©.

Bien que LuaTEX soit principalement un sur-ensemble de pdfTEX, quelques primitives utilitaires ont été supprimées (celles qui sont en quelque sorte remplacées par Lua) ou renommées. Ce paquetage les fournit avec des noms cohérents entre les différents moteurs, y compris XeTeX qui a récemment implémenté certaines de ces primitives, comme "\strcmp".
%

\pkdesc{magicnum}{\HO}{\luatex, \pdftex, \xetex}{\latex, Plain}{%
  macros/latex/contrib/oberdiek/}
Provides hierarchical access to ``magic numbers'' such as catcodes, group
types, etc. used internally by \tex and its successors. Under \luatex, a more
efficient implementation is used and a Lua interface is provided.

Fournit un accès hiérarchique aux "nombres magiques" tels que les "catcodes", les types de groupes, etc. utilisés en interne par TeX et ses successeurs. Sous LuaTeX, une implémentation plus efficace est utilisée et une interface Lua est fournie.
%%

\pkdesc{lua-alt-getopt}{Aleksey Cheusov}{\cmd{texlua}}{command-line}{%
  support/lua/lua-alt-getopt}[https://github.com/LuaDist/alt-getopt]
Command-line option parser, mostly compatible with POSIX and GNU getopt, to be
used in command-line Lua scripts such as \cmd{mkluatexfontdb} from
\pk{luaotfload}.

Parseur d'options de ligne de commande, principalement compatible avec POSIX et GNU "getopt", à utiliser dans les scripts Lua en ligne de commande tels que "mkluatexfontdb" ou "luaotfload".
%%

\section{Les formats \cmd{luatex} et \cmd{lualatex}}\label{formats}

This section is for developers and curious users only; normal users can safely
skip it. The following information apply to \texlive 2010, and most likely to
\miktex 2.9 too, though I didn't actually check. Earlier versions of \texlive
had slightly different and less complete arrangements.

Cette section est réservée aux développeurs et aux utilisateurs curieux ; les utilisateurs normaux peuvent la sauter. Les informations suivantes s'appliquent à TEX Live 2010, et très probablement à MikTEX 2.9 aussi, bien que je n'aie pas vérifié. Les versions antérieures de TeX Live avaient des dispositions légèrement différentes et moins complètes.
%%

\para{Primitive names}
As mentioned in section~\ref{things}, the names of the \luatex-specific
primitives are not the same in the \cmd{lualatex} format as in the \luatex
manual. In the \cmd{luatex} format (that is, \luatex with the Plain format),
primitives are available with their natural name, but also with the prefixed
name, in order to ease development of generic packages.

Comme mentionné dans la section 1.4, les noms des primitives spécifiques à LuaTeX ne sont pas les mêmes dans le format luatex que dans le manuel LuaTeX. Dans le format luatex (c'est-à-dire LuaTeX avec le format Plain), les primitives sont disponibles avec leur nom naturel, mais aussi avec le nom préfixé, afin de faciliter le développement de packages génériques.
%%

The rationale, copy-pasted from the file \file{lualatexiniconfig.tex} that
implements this for the \cmd{lualatex} format, is:

Le raisonnement, copié-collé du fichier "lualatexiniconfig.tex" qui implémente ceci pour le format lualatex, est le suivant:
%%

\begin{myquote}
  \begin{enumerate}
    \item All current macro packages run smoothly on top of pdf(e)TeX, so
      those primitives are left untouched.
    \item Other non-TeX82 primitives in \luatex may cause name clashes with
      existing macros in macro packages, especially when they use very
      ``natural'' names such as ©\outputbox©, ©\mathstyle© etc. Such a
      probability for name clashes is undesirable, since the most existing
      LaTeX documents that run without change under \luatex, the better.
    \item The \luatex team doesn't want to apply a systematic prefixing policy,
      but kindly provided a tool allowing prefixes to be applied. So we chose
      to use it.  Previously, we even disabled the extra primitives, but now
      we feel it's better to enable them with systematic prefixing, in order
      to avoid that each and every macro package (or user) enables them with
      various and inconsistent prefixes (including the empty prefix).
    \item The ©luatex© prefix was chosen since it is already used as a prefix
      for some primitives, such as ©\luatexversion©: this way, those
      primitives don't end up with a double prefix (for details, see
      ©tex.enableprimitives© in the \luatex manual).
    \item The ©\directlua© primitive is provided both with its natural name
      (allowing easy detection of \luatex) and a prefixed version
      ©\luatexdirectlua© (for consistency with ©\luatexlatelua©).
    \item Various remarks:
      \begin{itemize}
        \item The obvious drawback of such a prefixing policy is that the
          names used by \latex or generic macro writer won't match the names
          used in the manual.  We hope this is compensated by the gain in
          backwards compatibility.
        \item All primitives dealing with Unicode math already begin with ©\U©,
          and maybe will match the names of \xetex primitives some day, so
          maybe prefixing was not necessary/desirable for them. However, we
          tried to make the prefixing rule as simple as possible, so that
          the previous point doesn't get even worse.
        \item Maybe some day we'll feel it's better to provide all primitives
          without prefixing. If this happens, it will be easy to add the
          unprefixed primitives in the format while keeping the prefixed names
          for compatibility. It wouldn't work the other way round; i.e.,
          belatedly realizing that we should not provide the unprefixed
          primitives would then break any \luatex-specific macro packages
          that had been written.
      \end{itemize}
  \end{enumerate}
\end{myquote}

1. Tous les packages de macros actuels fonctionnent sans problème par dessus pdf(e)TeX, donc ces primitives sont conservées telles quelles.

2. D'autres primitives non-TeX82 dans LuaTeX peuvent provoquer des conflits de noms avec des macros existantes dans les packages de macros, en particulier lorsqu'elles utilisent des noms très "naturels" tels que "\outputbox", "\mathstyle", etc. La probabilité d'un tel conflit de noms est importante et ce n'est pas souhaitable, car plus les documents LaTeX existants fonctionneront dans leur état actuel sous LuaTeX, mieux ce sera.

3. L'équipe LuaTeX ne souhaite pas appliquer une politique de préfixage systématique, mais a gentiment fourni un outil permettant d'appliquer des préfixes. Nous avons donc choisi de l'utiliser. Auparavant, nous avions même désactivé les primitives supplémentaires, mais maintenant nous pensons qu'il est préférable de les activer avec un préfixe systématique, afin d'éviter que chaque package (ou chaque utilisateur) les active avec des préfixes variés et incohérents (y compris avec un préfixe vide).

4. Le préfixe "luatex" a été choisi car il est déjà utilisé comme préfixe pour certaines primitives, comme "\luatexversion" : de cette façon, ces primitives ne se retrouvent pas avec un double préfixe (pour plus de détails, voir "tex.enableprimitives" dans le manuel LuaTeX).

5. La primitive "\directlua" est fournie à la fois avec son nom naturel (permettant une détection facile de LuaTeX) et une version préfixée "\luatexdirectlua" (par souci de cohérence avec "\luatexlatelua").

6. Remarques diverses :

- L'inconvénient évident d'une telle politique de préfixage est que les noms utilisés par LaTeX ou l'outil d'écriture de macros génériques ne correspondront pas aux noms utilisés dans le manuel. Nous espérons que cet inconvénient est compensé par le gain en compatibilité ascendante.

- Toutes les primitives traitant des mathématiques Unicode commencent déjà par "\U", et correspondront peut-être aux noms des primitives XeTeX. les noms des primitives XeTeX un jour, donc peut-être que la préfixation n'était ni nécessaire ni souhaitable pour elles. Cependant, nous avons essayé de rendre la règle de préfixage aussi simple que possible, afin de ne pas aggraver le point précédent.

- Peut-être qu'un jour nous penserons qu'il est préférable de fournir toutes les primitives sans préfixe. Si cela se produit, il sera facile d'ajouter les primitives non préfixées dans le format tout en conservant les noms préfixés pour la compatibilité. Cela ne fonctionnerait pas dans l'autre sens ; c'est-à-dire que si l'on réalisait tardivement que nous ne devrions pas fournir de primitives sans préfixe, cela casserait tous les packages spécifiques à LuaTeX qui auront été écrits.
%%

\para{\cs{jobname}}[jobname]
The \latex kernel (and a lot of packages) use constructs like
©\input\jobname.aux© for various purposes. When ©\jobname© contains spaces,
this doesn't do the right thing, since the argument of ©\input© ends at the
first space. To work around this, \pdftex automagically quotes ©\jobname© when
needed, but \luatex doesn't for some reason. A nearly complete workaround is
included in \latex-based (as opposed to Plain-based) \luatex formats.

Le noyau LaTeX (et de nombreux packages) utilise des constructions comme "\input\jobname.aux" à diverses fins. Lorsque "\jobname" contient des espaces, cela ne fait pas ce qu'il faut, puisque l'argument de "\input" se termine au premier espace. Pour contourner ce problème, pdfTeX met automatiquement des guillemets autour de "\jobname" lorsque cela est nécessaire, mais LuaTeX ne le fait pas pour une raison inconnue. Une solution de contournement presque complète est incluse dans les formats LuaTeX basés sur LaTeX (par opposition à Plain).
%%

It doesn't work, however, if \luatex is invoked as ©lualatex '\input name'©,
as opposed to the more usual ©lualatex name©. To work around this
limitation of the workaround included in the format, specifying a jobname
explicitly, as in ©lualatex jobname=name '\input name'©. Or even better, just
don't use spaces in the names of your \tex files.


Elle ne fonctionne pas, cependant, si LuaTeX est invoqué en tant que "lualatex '\input name'", par opposition au plus habituel "lualatex name". Pour contourner cette limitation de la solution de contournement incluse dans le format, spécifiez un nom de travail explicitement, comme dans "lualatex jobname=name '\input name'". Ou encore mieux, n'utilisez pas d'espaces dans les noms de vos fichiers TeX.
%%

For more details, see
\href{http://www.ntg.nl/pipermail/dev-luatex/2009-April/002549.html}{this old
  thread} and
\href{http://tug.org/pipermail/luatex/2010-August/001986.html}{this newer one}
on the \luatex mailing lists, and \file{lualatexquotejobname.tex} for the
implementation of the workaround.

Pour plus de détails, voir cet ancien fil de discussion et ce fil plus récent sur les listes de diffusion LuaTeX, et "lualatexquotejobname.tex" pour l'implémentation de la solution de contournement.
%%

\para{hyphenation}
\luatex offers dynamic loading for hyphenation patterns. The support for this in
\pf{babel} and \pf{polyglossia} appeared only on \texlive 2013, but should
work well since then.

LuaTeX permet le chargement dynamique des motifs de césure. Le support pour ceci dans "babel" et "polyglossia" est apparu seulement sur TeX Live 2013, mais devrait bien fonctionner depuis.
%%

Documentation and implementation details are included in
\file{luatex-hyphen.pdf}. The sources are part of the
\href{http://tug.org/tex-hyphen/}{texhyphen project}.

La documentation et les détails d'implémentation sont inclus dans "luatex-hyphen.pdf". Les sources font partie du projet "texhyphen".
%%

\para{codes}
The engine itself does not set ©\catcode©s, ©\lccode©s, etc. for non-ASCII
characters. Correct ©\lccode©s in particular are essential for hyphenation to
work. Formats for \luatex now include \file{luatex-unicode-letters.tex}, a
modified version of \file{unicode-letters.tex} from the \xetex distribution,
that takes care of settings these values in accordance with the Unicode
standard.

Le moteur lui-même ne définit pas les "\catcode "s, "\lccode "s, etc. pour les caractères non-ASCII. Des "\lccode "s corrects, en particulier, sont essentiels pour que la césure fonctionne. Les formats pour LuaTeX incluent maintenant "luatex-unicode-letters.tex", une version modifiée de "unicode-letters.tex" de la distribution XeTeX, qui se charge de paramétrer ces valeurs conformément à la norme Unicode.
%%

This was added after \texlive 2010 went out, so you are strongly advised to
update your installation if you want to enjoy proper hyphenation for non-ASCII
text.

Ceci a été ajouté après la sortie de TeX Live 2010, il est donc fortement conseillé de mettre à jour votre installation si vous voulez profiter d'une césure correcte pour les textes non-ASCII.
%%


\section{Things that just work, partially work, or don't work (yet)}
\label{workornot}

\subsection{Just working}\label{working}

\para{Unicode}
Conventional \latex offers some level of support for UTF-8 in input files.
However, at a low level, non-ASCII characters are not atomic in this case:
they consist of several elementary pieces (known as \emph{tokens} to
\tex{}nicians). Hence, some packages that scan text character by character or
do other atomic operations on characters (such as changing their catcodes)
often have problems with UTF-8 in conventional \latex. For example, you can't
use any non-ASCII character for short verbatim with \pf{fancyvrb}, etc.

LaTeX conventionnel offre un certain niveau de support pour UTF-8 dans les fichiers d'entrée. Cependant, à bas niveau, les caractères non-ASCII ne sont pas atomiques dans ce cas : ils sont constitués de plusieurs morceaux élémentaires (connus sous le nom de tokens par les TeXniciens). Par conséquent, certains packages qui analysent le texte caractère par caractère ou effectuent d'autres opérations atomiques sur les caractères (comme la modification de leurs "catcodes") ont souvent des problèmes avec UTF-8 en LaTeX conventionnel. Par exemple, vous ne pouvez pas utiliser n'importe quel caractère non-ASCII pour le verbatim court avec "fancyvrb", etc.
%%

The good news is, with \lualatex, some of these package's features start
working on arbitrary Unicode characters without needing to modify the package.
The bad news is, this isn't always true. See the next section for details.

La bonne nouvelle est qu'avec LuaLaTeX, certaines des fonctionnalités de ces paquets commencent à fonctionner sur des caractères Unicode arbitraires sans avoir besoin de modifier le package. La mauvaise nouvelle est que ce n'est pas toujours vrai. Voir la section suivante pour plus de détails.
%%

\subsection{Partially working}\label{partial}

\para{microtype}
Package \pf{microtype} has limited support for \luatex: more precisely, as of
version 2.5 2013/03/13, protrusion and expansion are available and activated
by default in PDF mode, but kerning, spacing and tracking are not supported
(see table~1 in section~3.1 of \file{microtype.pdf}).

Le package "microtype" n'est pas entièrement compatible avec LuaTeX : plus précisément, à partir de la version 2.5 2013/03/13, la protrusion et l'expansion sont disponibles et activées par défaut en mode PDF, mais le crénage, l'espacement et le suivi ne sont pas supportés (voir tableau 1 dans la section 3.1 de "microtype.pdf").
%%

On the other hand, \pk{luaotfload}, loaded by \pk{fontspec}, supports a lot of
microtypographic features. So the only problem is the lack of a unified
interface.

D'un autre côté, "luaotfload", chargé par "fontspec", supporte un grand nombre de fonctionnalités microtypographiques. Le seul problème est donc l'absence d'une interface unifiée.
%%

\para{xunicode}
Package \pf{xunicode}'s main feature is to ensure that the usual control
sequences for non-ASCII characters (such as ©\'e©) do the right thing in a
Unicode context. It could \emph{probably} work with \luatex, but explicitly
checks for \xetex only. However, \pk{fontspec} uses a trick to load it anyway.
So, you can't load it explicitly, but you don't need to, since \pk{fontspec}
already took care of it.

La principale fonctionnalité du package "xunicode" est de s'assurer que les séquences de contrôle habituelles pour les caractères non-ASCII (comme "\'e") font ce qu'il faut dans un contexte Unicode. Il pourrait probablement fonctionner avec LuaTeX, mais ne vérifie explicitement que pour XeTeX. Cependant, "fontspec" utilise une astuce pour le charger de toutes façons. Donc, vous ne pouvez pas le charger explicitement, mais vous n'en avez pas besoin, puisque "fontspec" s'en est déjà occupé.
%%

\para{encodings}
As mentioned in the above section, a few things that were problematic with
UTF-8 on conventional \latex spontaneously works, but not always. For example,
with the \pf{listings} package on \lualatex, you may use only characters below
256 (that is, characters from the Latin-1 set), inside your listings (but of
course the full Unicode range is still available in the rest of your
document).

Comme mentionné dans la section précédente, certaines choses qui posaient problème avec UTF-8 sur LaTeX conventionnel fonctionnent maintenant naturellement, mais pas toujours. Par exemple, avec le package listings de LuaLaTeX, vous ne pouvez utiliser que les caractères inférieurs à 256 (c'est-à-dire les caractères du jeu Latin-1), dans vos listings (mais bien sûr, toute la gamme Unicode est toujours disponible dans le reste de votre document).
%%

\para{metrics}
This item isn't exactly about working or not working, but rather about not
working in exactly the same way as \pdftex or \xetex: you may observe minor
differences in the layout and hyphenation of your text.

Ce point ne parle pas exactement du fait que cela fonctionne ou non, mais plutôt du fait que cela ne fonctionne pas exactement de la même manière que pdfTeX ou XeTeX : vous pouvez observer des différences mineures dans la mise en page et la césure de votre texte.
%%

They may be due variations between two versions of the same font used by the
various engines, adjustments made to the hyphenation, ligaturing or kerning
algorithms (or example, the first word of a paragraph, as well as words
containing different fonts, can now be hyphenated), or differences in the
hyphenation patterns used (patterns used by \pdftex are basically frozen, but
\luatex and \xetex use newer version for some languages) for this language.

Elles peuvent être dues à des variations entre deux versions de la même police utilisée par les différents moteurs, à des ajustements apportés aux algorithmes de césure, de ligature ou de crénage (par exemple, le premier mot d'un paragraphe, ainsi que les mots contenant des polices différentes, peuvent désormais être césurés), ou à des différences dans les motifs de césure utilisés (les motifs utilisés par pdfTeX sont fondamentalement figés, mais LuaTeX et XeTeX utilisent des versions plus récentes pour certaines langues) pour cette langue.
%%

If you ever observe a major difference between pdf\latex and \lualatex with
the same fonts, it is not at all unlikely that a bug in \luatex\footnote{For
  example, \luatex 0.60 had a bug that prevented any hyphenation after a
  \code{-{}-{}-} ligature until the end of the paragraph.} or in the font is
involved. As usual, make sure your distribution is up-to-date before reporting
such a problem.

Si vous observez une différence majeure entre pdfLaTeX et LuaLaTeX avec les mêmes polices, il n'est pas du tout improbable qu'un bug dans LuaTeX (par exemple, LuaTeX 0.60 avait un bug qui empêchait toute césure après une ligature --- jusqu'à la fin du paragraphe) ou dans la police soit impliqué. Comme d'habitude, assurez-vous que votre distribution est à jour avant de signaler un tel problème.
%%

\para{babel}
Mostly working mostly without problems for Latin languages. For other
languages, your mileage may vary. Even for Latin languages, encoding-related
problems my happen.


Fonctionne pour l'essentiel sans problème pour les langues latines. Pour les autres langues, les résultats peuvent varier. Même pour les langues latines, des problèmes liés à l'encodage peuvent survenir.
%%

\para{polyglossia}
A more modern, but less complete, package for multilingual support,
\pf{polyglossia}, is also available and should be preffered, though it does
not support all \pf{babel} features yet.

Un package plus moderne, mais moins complet, pour le support multilingue, "polyglossia", est également disponible et devrait être préféré, bien qu'il ne supporte pas encore toutes les fonctionnalités de "babel".
%%

\subsection{Not working (yet)}\label{notworking}

\para{old encodings}[oldenco] Packages playing with input (source files) or
output (fonts) encodings are very likely to break with \luatex. This includes
\pf{inputenc}, \pf{fontenc}, \pf{textcomp}, and probably most classical font
packages such as \pf{mathptmx} or \pf{fourier}. The good news
is, Unicode is a more powerful way to handle encoding problems that old
packages were trying to solve, so you most likely don't need these packages
anyway. However, not everything is already ported to the shiny new world of
Unicode, so you may have a more limited (or just different) set of choices
available for some time (this is especially true for fonts).

Les packages jouant avec les encodages d'entrée (fichiers sources) ou de sortie (polices) sont très susceptibles de ne pas fonctionner avec LuaTeX. Cela inclut "inputenc", "fontenc", "textcomp", et probablement la plupart des packages de polices classiques tels que "mathptmx" ou "fourier". La bonne nouvelle est qu'Unicode est un moyen plus puissant de gérer les problèmes d'encodage que les anciens paquets essayaient de résoudre, donc vous n'avez probablement plus besoin de ces paquets. Cependant, tout n'a pas encore été porté dans le nouveau monde merveilleux d'Unicode, et il se peut que vous ayez un choix plus limité (ou simplement différent) pendant un certain temps (ceci est particulièrement vrai pour les polices).
%%

\para{spaces} Spaces in file names are not really well supported in the \tex
world in general. This doesn't really get better with \luatex. Also, due to
tricky reasons, things may be worse if you have spaces in the name of your main
\tex file \emph{and} don't invoke \luatex in the usual way. If you do
invoke it in the usual way, everything should work, and I won't tell you what
the unusual invocation looks like. Otherwise, read the point about
\pararef{jobname} in section~\ref{formats} for a workaround and technical
details. Or even better, don't use spaces in the names of your \tex files.

Les espaces dans les noms de fichiers ne sont pas vraiment bien supportés dans le monde TeX en général. Cela ne s'améliore pas vraiment avec LuaTeX. De plus, pour des raisons délicates, les choses peuvent être pires si vous avez des espaces dans le nom de votre fichier TeX principal et que vous n'invoquez pas LuaTeX de la manière habituelle. Si vous l'invoquez de la manière habituelle, tout devrait fonctionner, et je ne vous dirai pas à quoi ressemble l'invocation inhabituelle. Sinon, lisez le point sur "jobname" dans la section 4 pour une solution de contournement et des détails techniques. Ou encore mieux, n'utilisez pas d'espaces dans les noms de vos fichiers TeX.
%%

\end{document}

% vim: spell spelllang=en
